%\documentclass[11pt]{extarticle} %extarticle for fontsizes other than 10, 11 And 12
\documentclass[11p]{article}

%%%%%%%%%%%%%%%%%%%%%%%% Packages %%%%%%%%%%%%%%%%%%%%%%%%
\usepackage{amscd}
\usepackage{amsmath}
\usepackage{amssymb}
\usepackage{amsthm}
\usepackage{amsxtra}
\usepackage{bbold}
%\usepackage{bigints}
\usepackage{color}
\usepackage{dsfont}
\usepackage{enumerate}
\usepackage[mathscr]{eucal}
%\usepackage{fancyhdr}
\usepackage{float}
%\usepackage{fullpage} %% Dont use this for beamer presentations
\usepackage{geometry}
\usepackage{graphicx}
\usepackage{hyperref}
\usepackage{indentfirst}
\usepackage{latexsym}
\usepackage{listings}
\usepackage{lscape}
%\usepackage{mathtools}
\usepackage{microtype}
\usepackage{natbib}
\usepackage{pdfpages}
\usepackage{verbatim}
\usepackage{wrapfig}
\usepackage{xargs}
\DeclareGraphicsExtensions{.pdf,.png,.jpg, .jpeg}

%%%%%%%%%%%%%%%%%%%%%%%% Commands %%%%%%%%%%%%%%%%%%%%%%%%
\newcommand{\Sup}{\textsuperscript}
\newcommand{\Exp}{\mathds{E}}
\newcommand{\Prob}{\mathds{P}}
\newcommand{\Z}{\mathds{Z}}
\newcommand{\Ind}{\mathds{1}}
\newcommand{\A}{\mathcal{A}}
\newcommand{\F}{\mathcal{F}}
\newcommand{\G}{\mathcal{G}}
\newcommand{\I}{\mathcal{I}}
\newcommand{\be}{\begin{equation}}
\newcommand{\ee}{\end{equation}}
\newcommand{\bes}{\begin{equation*}}
\newcommand{\ees}{\end{equation*}}
\newcommand{\union}{\bigcup}
\newcommand{\intersect}{\bigcap}
\newcommand{\Ybar}{\overline{Y}}
\newcommand{\ybar}{\bar{y}}
\newcommand{\Xbar}{\overline{X}}
\newcommand{\xbar}{\bar{x}}
\newcommand{\betahat}{\hat{\beta}}
\newcommand{\Yhat}{\widehat{Y}}
\newcommand{\yhat}{\hat{y}}
\newcommand{\Xhat}{\widehat{X}}
\newcommand{\xhat}{\hat{x}}
\newcommand{\E}[1]{\operatorname{E}\left[ #1 \right]}
%\newcommand{\Var}[1]{\operatorname{Var}\left( #1 \right)}
\newcommand{\Var}{\operatorname{Var}}
\newcommand{\Cov}[2]{\operatorname{Cov}\left( #1,#2 \right)}
\newcommand{\N}[2][1=\mu, 2=\sigma^2]{\operatorname{N}\left( #1,#2 \right)}
\newcommand{\bp}[1]{\left( #1 \right)}
\newcommand{\bsb}[1]{\left[ #1 \right]}
\newcommand{\bcb}[1]{\left\{ #1 \right\}}
\newcommand*{\permcomb}[4][0mu]{{{}^{#3}\mkern#1#2_{#4}}}
\newcommand*{\perm}[1][-3mu]{\permcomb[#1]{P}}
\newcommand*{\comb}[1][-1mu]{\permcomb[#1]{C}}

%%%%%%%%%%%%%%%%%%% To change the margins and stuff %%%%%%%%%%%%%%%%%%%
\geometry{left=1in, right=1in, top=1in, bottom=0.8in}
%\setlength{\voffset}{0.5in}
%\setlength{\hoffset}{-0.4in}
%\setlength{\textwidth}{7.6in}
%\setlength{\textheight}{10in}
%%%%%%%%%%%%%%%%%%%%%%%%%%%%%%%%%%%%%%%%%%%%%%%%%%%%%%%%%%%%%%%%%%%%%%%

\usepackage{Sweave}
\begin{document}
\input{Update_2014-11-30-concordance}

\bibliographystyle{plain}  %Choose a bibliograhpic style

\title{Comparing different clustering techniques in analyzing gene-expression data}
\author{Subhrangshu Nandi\\
  Stat 760: Multivariate Analysis\\
  Project Report \\
  nandi@stat.wisc.edu}
\date{November 25, 2014}
%\date{}

%\maketitle

\begin{center}
{\Large{Comparing different clustering techniques in analyzing gene-expression data}}\\
%Subhrangshu Nandi\\
%Stat 760: Multivariate Analysis\\
%Project Proposal\\
November 30, 2014
\end{center}

\noindent
The results of using the {\emph{clValid package}} in R, on Data 2 are:

\noindent
For {\emph{Gaussian Mixture Model clustering}}\\
\begin{Schunk}
\begin{Soutput}
[1] "Mclust recommends:"
\end{Soutput}
\begin{Soutput}
[1] "VVV ( ellipsoidal, varying volume, shape, and orientation ) model with 6 components"
\end{Soutput}
\end{Schunk}
\\
\noindent
For {\emph{Hierarchical clustering}}\\
% latex table generated in R 3.1.2 by xtable 1.7-4 package
% Sun Nov 30 22:22:32 2014
\begin{table}[ht]
\centering
\begin{tabular}{rrrrrr}
  \hline
 & 4 & 5 & 6 & 7 & 8 \\ 
  \hline
Connectivity & 12.9972 & 16.7996 & 22.2563 & 22.7063 & 26.0865 \\ 
  Dunn & 0.4779 & 0.5717 & 0.6107 & 0.6107 & 0.6107 \\ 
  Silhouette & 0.1748 & 0.2481 & 0.2405 & 0.2344 & 0.2290 \\ 
   \hline
\end{tabular}
\end{table}\\
For {\emph{k-means clustering}}\\
% latex table generated in R 3.1.2 by xtable 1.7-4 package
% Sun Nov 30 22:22:33 2014
\begin{table}[ht]
\centering
\begin{tabular}{rrrrrr}
  \hline
 & 4 & 5 & 6 & 7 & 8 \\ 
  \hline
Connectivity & 22.2528 & 26.5933 & 31.5496 & 32.8952 & 30.3833 \\ 
  Dunn & 0.5710 & 0.5130 & 0.5269 & 0.5157 & 0.5792 \\ 
  Silhouette & 0.2562 & 0.2168 & 0.2024 & 0.1800 & 0.2086 \\ 
   \hline
\end{tabular}
\end{table}\\
For {\emph{PAM clustering}}\\
% latex table generated in R 3.1.2 by xtable 1.7-4 package
% Sun Nov 30 22:22:33 2014
\begin{table}[ht]
\centering
\begin{tabular}{rrrrrr}
  \hline
 & 4 & 5 & 6 & 7 & 8 \\ 
  \hline
Connectivity & 25.1405 & 26.5337 & 32.0750 & 35.1302 & 36.4980 \\ 
  Dunn & 0.3658 & 0.3658 & 0.3658 & 0.3712 & 0.3712 \\ 
  Silhouette & 0.1458 & 0.1577 & 0.1548 & 0.1512 & 0.1601 \\ 
   \hline
\end{tabular}
\end{table}\\
Final recommendations:
\begin{table}[H]
\centering
\begin{tabular}{lrll}
\hline
& Score & Method & Clusters \\
\hline
Connectivity & 12.9972 & hierarchical & 4 \\
Dunn & 0.6107 & hierarchical & 6 \\
Silhouette & 0.2562 & kmeans & 4 \\
\hline
\end{tabular}
\end{table}


\newpage
\bibliography{Reference_Stat760_ProjectReport}
%\bibliography{research}

\end{document}
