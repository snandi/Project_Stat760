%\documentclass[10pt,dvipsnames,table, handout]{beamer} % To printout the slides without the animations
\documentclass[10pt,dvipsnames,table]{beamer} 
%\usetheme{Luebeck} 
\usetheme{Madrid} 
%\usetheme{Marburg} 
\setbeamercolor{structure}{fg=cyan!90!white}
\setbeamercolor{normal text}{fg=white, bg=black}

%%%%%%%%%%%%%%%%%%%%%%%% Packages %%%%%%%%%%%%%%%%%%%%%%%%
\usepackage{amscd}
\usepackage{amsmath}
\usepackage{amssymb}
\usepackage{amsthm}
\usepackage{amsxtra}
\usepackage{bbold}
%\usepackage{bigints}
\usepackage{color}
\usepackage{dsfont}
\usepackage{enumerate}
\usepackage[mathscr]{eucal}
%\usepackage{fancyhdr}
\usepackage{float}
%\usepackage{fullpage} %% Dont use this for beamer presentations
\usepackage{geometry}
\usepackage{graphicx}
\usepackage{hyperref}
\usepackage{indentfirst}
\usepackage{latexsym}
\usepackage{listings}
\usepackage{lscape}
\usepackage{mathtools}
\usepackage{microtype}
\usepackage{natbib}
\usepackage{pdfpages}
\usepackage{verbatim}
\usepackage{wrapfig}
\usepackage{xargs}
\DeclareGraphicsExtensions{.pdf,.png,.jpg, .jpeg}

%%%%%%%%%%%%%%%%%%%%%%%% Commands %%%%%%%%%%%%%%%%%%%%%%%%
\newcommand{\Sup}{\textsuperscript}
\newcommand{\Exp}{\mathds{E}}
\newcommand{\Prob}{\mathds{P}}
\newcommand{\Z}{\mathds{Z}}
\newcommand{\Ind}{\mathds{1}}
\newcommand{\A}{\mathcal{A}}
\newcommand{\F}{\mathcal{F}}
\newcommand{\G}{\mathcal{G}}
\newcommand{\I}{\mathcal{I}}
\newcommand{\be}{\begin{equation}}
\newcommand{\ee}{\end{equation}}
\newcommand{\bes}{\begin{equation*}}
\newcommand{\ees}{\end{equation*}}
\newcommand{\union}{\bigcup}
\newcommand{\intersect}{\bigcap}
\newcommand{\Ybar}{\overline{Y}}
\newcommand{\ybar}{\bar{y}}
\newcommand{\Xbar}{\overline{X}}
\newcommand{\xbar}{\bar{x}}
\newcommand{\betahat}{\hat{\beta}}
\newcommand{\Yhat}{\widehat{Y}}
\newcommand{\yhat}{\hat{y}}
\newcommand{\Xhat}{\widehat{X}}
\newcommand{\xhat}{\hat{x}}
\newcommand{\E}[1]{\operatorname{E}\left[ #1 \right]}
%\newcommand{\Var}[1]{\operatorname{Var}\left( #1 \right)}
\newcommand{\Var}{\operatorname{Var}}
\newcommand{\Cov}[2]{\operatorname{Cov}\left( #1,#2 \right)}
\newcommand{\N}[2][1=\mu, 2=\sigma^2]{\operatorname{N}\left( #1,#2 \right)}
\newcommand{\bp}[1]{\left( #1 \right)}
\newcommand{\bsb}[1]{\left[ #1 \right]}
\newcommand{\bcb}[1]{\left\{ #1 \right\}}
\newcommand*{\permcomb}[4][0mu]{{{}^{#3}\mkern#1#2_{#4}}}
\newcommand*{\perm}[1][-3mu]{\permcomb[#1]{P}}
\newcommand*{\comb}[1][-1mu]{\permcomb[#1]{C}}


%%%%%%%%%%%%% For explanatory bubbles, use the following code %%%%%%%%%%%%%
%% \usepackage{tikz} %% For explanatory bubbles
%% \usepackage{xparse}
%% \usetikzlibrary{shapes.callouts,ocgx}

%% \newcommand{\tikzmark}[1]{\tikz[overlay,remember picture,baseline=0.5ex] \node (#1) {};}

%% % \explainword: #1= identifier to mark the word, #2 text
%% \NewDocumentCommand{\explainword}{r[] m}{
%%     \switchocg{#1}{#2}\tikzmark{#1}
%% }

%% \tikzset{my callout style/.style={
%%         draw,rectangle callout,anchor=pointer,callout relative pointer={(230:1cm)},
%%         rounded corners,align=center,text width=2cm,fill=cyan!20, 
%%     }
%% }

%% % \mycallout: #1 opacity style, #2 pointer base position, #3= text
%% \NewDocumentCommand{\mycallout}{O{opacity=0.8,text opacity=1} m m}{%
%% \begin{tikzpicture}[remember picture, overlay]
%%  \begin{scope}[ocg={ref=#2,status=invisible,name={#3}}]
%% \node[my callout style,#1]at (#2) {#3};
%% \end{scope}
%% \end{tikzpicture}
%% }
%%%%%%%%%%%%%%%%%%%%%%%%%%%%%%%%%%%%%%%%%%%%%%%%%%%%%%%%%%%%%%%%%

%%%%%%%%%%%%%%%%%%%%%%%% TITLE PAGE %%%%%%%%%%%%%%%%%%%%%%%%
\DeclarePairedDelimiter\ceil{\lceil}{\rceil}
\title[Clustering gene-expression data]{Comparing different clustering techniques in analyzing gene-expression data}
\author{Subhrangshu Nandi}
\institute[Stat 760]{Stat 760, Fall 2014 \\
  Department of Statistics \\
 University of Wisconsin-Madison}
\date{December 11, 2014}

\begin{document}
\setlength{\baselineskip}{16truept}
\frame{\maketitle}

%%%%%%%%%%%% Slide 1 %%%%%%%%%%%%
%\begin{frame}
%\frametitle{Outline}
%\begin{itemize}
%\pause \item Introduction to functional data
%\pause \item Motivation of potential problem
%\pause \item Objective: Minimize effets of smoothing in two sample tests
%\pause \item Problem definition and two sample test
%\pause \item Some theoretical results
%\pause \item Computational results
%\pause \item Summary
%\end{itemize}
%\end{frame}

%%%%%%%%%%%% Slide 2 %%%%%%%%%%%%
\begin{frame}
\frametitle{Objective}
\begin{enumerate}
\pause \item Learn different clustering techniques
\begin{itemize}
\item K-means
%\item K-mediods (or Partitioning around \explainword[mediods]{mediods} - PAM)
\item K-mediods (or Partitioning around mediods - PAM)
\item Hierarchical
\item Gaussian mixture modeling
\end{itemize}
\pause \item Learn different metrics to compare clusters
\begin{itemize}
\item Connectivity index
\item Dunn index
\item Silhouette width
\end{itemize}
\pause \item Apply to gene-expression data and evaluate different clusters using these indices
\end{enumerate}

%% \mycallout{mediods}{Always members of the dataset}
%% \tikzset{my callout style/.append style={fill=orange!20}}

\end{frame}
%%%%%%%%%%%%%%%%%%%%%%%%%%%%%%%%%




% \begin{frame}
% \begin{theorem}
% (i) Let $m=1,\ldots,s_1.$ We consider two runs $C_i,C_j,i\neq j$ in the subdesign $C^{m}.$ Then for $k=1,\ldots,d$ and $x,y\in \textbf{Z}_{s_1^2}$ 
% $$
% Pr(C_{ik}=x, C_{jk}=y) =\begin{cases}
% \frac{1}{s_1^3(s_1-1)},& \ceil{\frac{x}{s_1}}\neq\ceil{\frac{y}{s_1}}\\
% 0,&o.w.
% \end{cases}
% $$
% (ii) The joint probability mass function for $c_{m_1,ik},c_{m_2,jk},m_1\neq m_2$ is 
% $$
% Pr(c_{m_1,ik}=x,c_{m_2,jk}=y) =\begin{cases}
% \frac{1}{s_1^4},& \ceil{\frac{x}{s_1}}\neq\ceil{\frac{y}{s_1}}\\
% \frac{1}{s_1^3(s_1-1)},&\ceil{\frac{x}{s_1}}=\ceil{\frac{y}{s_1}},x\neq y\\
% 0,&o.w.
% \end{cases}
% $$
% \end{theorem}
% \end{frame}

% \begin{frame}
% \begin{theorem}
% Consider two row $C_i,C_j,i\neq j$ in the slice $C^{(m)}.$ Then for $k,l=1,\ldots,d$ and $k\neq l$
% $$ Pr(c_{ik},c_{il},c_{jk},c_{jl}) =\begin{cases}
% \frac{1}{s_1^2s_2(s_2-1)(s_1-1)},& (i,j)\in H_1\\
% \frac{1}{s_1^2s_2^2(s_1-1)},&(i,j)\in H_2.
% \end{cases}
% $$
% \end{theorem}
% \begin{theorem}
% \begin{itemize}
% \item $Var(\hat{\mu}_m) = s_1^{-1}\sum_{|u|\geq3}Var\{f_{m,u}(X)\} + o(s_1^{-1})$
% \item $Var(\hat{\mu}) = s_1^{-2}\sum_{|u|\geq3}Var\{f_u(X)\} + o(s_1^{-2})$
% \end{itemize}
% \end{theorem}
% \end{frame}
% %%%%%%%%%%%%%


\end{document}

